\section*{Conference Information}\addcontentsline{toc}{section}{Conference Information}

\addcontentsline{toc}{subsection}{Participants}

\noindent \textbf{Check-in and Registration:}  Early check-in and registration will be available Sunday, August 4 from 2pm to 8pm in the lobby of the Education Building at 35 W. 4th Street. On August 5-7 the registration desk will be located in the Kimmel Center lobby. On August 5, it will be open from 8am to 8pm; on August 6, it will be open from 8:30am to 5pm; on August 7, it will be open from 8:30am to noon.\\

\noindent \textbf{Wi-Fi:} Guest Wi-Fi access is available in the Kimmel Center.  The network password, which is changed weekly, will be available at the registration desk and posted on signs in all of the presentation spaces.  Conference attendees can also connect to the internet using eduroam if their home institution has enabled eduroam authentication (IdP). For more information on eduroam see \href{https://www.eduroam.org}{https://www.eduroam.org}.\\

\noindent \textbf{Social Media:} The hashtag for the conference is \#SMPC2019. To discuss a specific talk session, add the session code \#SMPC2019 \#E4 to help organize content. Feel free to link to abstracts posted on the website as needed. If you would like the tweet to be retweeted by the SMPC2019 account, please mention us @smpc2019.\\

\noindent \textbf{Lactation Room:} A lactation room will be made available upon request. Please speak to a staff member to coordinate. 

\subsection*{Talk Presentation Information}\addcontentsline{toc}{subsection}{Talk Presentations}
\noindent \textbf{Presentation Equipment:} You have the choice of using your own laptop or a Windows-based laptop in the presentation room. If you are not using your own laptop, you must bring your slides on a USB drive or have it accessible on the internet so it can been loaded onto the room machine prior to your talk. If you are using your own laptop, the available connections are both VGA and HDMI; please bring any adapters necessary for your machine. NOTE: the aspect ratio of all projectors in the Kimmel Center is 16:9; please format your presentations accordingly to prevent information from being obscured or other visual distortions.\\

\noindent \textbf{Presentation Setup:} All presenters must test their setup or upload their files to the room computer during one of the breaks prior to their session.  Presentation rooms will be available in the mornings starting at 8:30am.\\

\noindent \textbf{Presentation Timing and Chairing:} Each spoken presentation will have a session chair, who will introduce speakers by name, affiliation, and talk title. If you are not the primary author and are presenting, please let the chair know so that you can be introduced correctly. Talks are 12 minutes, with 3 minutes for questions and transition. The chair will communicate timing with the following: 
\begin{itemize}
\item 1 bell = 2 minutes left
\item 3 bells = time is up
\item ongoing bell ringing = you have used up even your Q\&A time and are about to eat into the next presentation. You are done.
\end{itemize}

\subsection*{Poster Presentation Information}\addcontentsline{toc}{subsection}{Poster Presentations}
All poster sessions take place on the 10th floor of the Kimmel Center, in the Rosenthal Pavilion. Those presenting the afternoon of Tuesday, August 6 must put up their posters in the designated locations between 1:00-3:00pm on August 6. Those presenting the morning of Wednesday, August 7 must put up their posters between 9:00-10:30am on August 7. Each posterboard space will be labeled, and the precise posting locations for each presenter will be available on-site at the registration desk and Rosenthal.


