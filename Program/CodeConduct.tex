\newpage
\section*{SMPC Code of Conduct}\addcontentsline{toc}{section}{SMPC Code of Conduct}


The Society for Music Perception and Cognition is dedicated to providing a harassment-free conference experience for everyone regardless of gender, gender identity and expression, sexual orientation, disability, physical appearance, body size, race, age or religion. We do not tolerate harassment of conference participants in any form. Sexual language and imagery is not appropriate for any conference venue, including talks. Conference participants violating these rules may be sanctioned or expelled from the conference at the discretion of the conference organizers. 

Harassment includes, but is not limited to:
\begin{itemize}
\item Verbal comments that reinforce social structures of domination (related to gender, gender identity and expression, sexual orientation, disability, physical appearance, body size, race, age, or religion)
\item Sexual images in public spaces
\item Deliberate intimidation, stalking, or following 
\item Harassing photography or recording
\item Sustained disruption of talks or other events
\item Inappropriate physical contact
\item Unwelcome sexual attention
\item Advocating for, or encouraging, any of the above behaviour
\end{itemize}

\subsubsection*{Enforcement}

Participants asked to stop any harassing behavior are expected to comply immediately. If a participant engages in harassing behaviour, event organizers retain the right to take any actions to keep the event a welcoming environment for all participants. This includes warning the offender or expulsion from the conference.

Event organizers may take action to redress anything designed to, or with the clear impact of, disrupting the event or making the environment hostile for any participants. We expect participants to follow these rules at all event venues and event-related social activities. We think people should follow these rules outside event activities too!

\subsubsection*{Reporting}

If someone makes you or anyone else feel unsafe or unwelcome, please report it as soon as possible. Harassment and other code of conduct violations reduce the value of the SMPC meeting for everyone. 

You can make a report either personally or anonymously.

\subsubsection*{Anonymous Report}

You can make an anonymous report by filling out the form at: \url{http://bit.ly/SMPC_report} 

We can't follow up an anonymous report with you directly, but we will fully investigate it and take whatever action is necessary to prevent a recurrence.

\subsubsection*{Personal Report}

You can make a personal report by emailing any of the SMPC Board members:
\begin{itemize}
\item Elizabeth Margulis (President): margulis@princeton.edu
\item Michael Schutz (Secretary): schutz@mcmaster.ca
\item Erin Hannon (Treasurer): erin.hannon@unlv.edu
\item Dominique Vuvan: d.vuvan@gmail.com
\item Amy Belfi: amybelfi@mst.edu 
\item Petr Janata: pjanata@ucdavis.edu 
\item Sarah Creel: screel@ucsd.edu
\item Bob Slevc: slevc@umd.edu 
\item Psyche Loui: p.loui@northeastern.edu 
\item David Baker (student representative): davidjohnbaker1@gmail.com 
\end{itemize}
When taking a personal report, we'll ask you to tell us about what happened. This can be upsetting, but you won't be asked to confront anyone and we won't tell anyone who you are.

SMPC leaders will be happy to help you contact hotel/venue security, local law enforcement, local support services, provide escorts, or otherwise assist you to feel safe for the duration of the event. We value your attendance.
